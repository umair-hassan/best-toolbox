%% Generated by Sphinx.
\def\sphinxdocclass{report}
\documentclass[letterpaper,10pt,english]{sphinxmanual}
\ifdefined\pdfpxdimen
   \let\sphinxpxdimen\pdfpxdimen\else\newdimen\sphinxpxdimen
\fi \sphinxpxdimen=.75bp\relax
\ifdefined\pdfimageresolution
    \pdfimageresolution= \numexpr \dimexpr1in\relax/\sphinxpxdimen\relax
\fi
%% let collapsable pdf bookmarks panel have high depth per default
\PassOptionsToPackage{bookmarksdepth=5}{hyperref}

\PassOptionsToPackage{warn}{textcomp}
\usepackage[utf8]{inputenc}
\ifdefined\DeclareUnicodeCharacter
% support both utf8 and utf8x syntaxes
  \ifdefined\DeclareUnicodeCharacterAsOptional
    \def\sphinxDUC#1{\DeclareUnicodeCharacter{"#1}}
  \else
    \let\sphinxDUC\DeclareUnicodeCharacter
  \fi
  \sphinxDUC{00A0}{\nobreakspace}
  \sphinxDUC{2500}{\sphinxunichar{2500}}
  \sphinxDUC{2502}{\sphinxunichar{2502}}
  \sphinxDUC{2514}{\sphinxunichar{2514}}
  \sphinxDUC{251C}{\sphinxunichar{251C}}
  \sphinxDUC{2572}{\textbackslash}
\fi
\usepackage{cmap}
\usepackage[T1]{fontenc}
\usepackage{amsmath,amssymb,amstext}
\usepackage{babel}



\usepackage{tgtermes}
\usepackage{tgheros}
\renewcommand{\ttdefault}{txtt}



\usepackage[Bjarne]{fncychap}
\usepackage{sphinx}

\fvset{fontsize=auto}
\usepackage{geometry}


% Include hyperref last.
\usepackage{hyperref}
% Fix anchor placement for figures with captions.
\usepackage{hypcap}% it must be loaded after hyperref.
% Set up styles of URL: it should be placed after hyperref.
\urlstyle{same}


\usepackage{sphinxmessages}




\title{BEST Toolbox Wiki}
\date{Jul 13, 2021}
\release{12072021}
\author{Umair Hassan}
\newcommand{\sphinxlogo}{\vbox{}}
\renewcommand{\releasename}{Release}
\makeindex
\begin{document}

\pagestyle{empty}
\sphinxmaketitle
\pagestyle{plain}
\sphinxtableofcontents
\pagestyle{normal}
\phantomsection\label{\detokenize{index::doc}}


\sphinxAtStartPar
Brain Electrophysiological recording and STimulation (BEST) Toolbox, is a MATLAB based open source software that interfaces with a wide variety of EEG, EMG, TMS and other stimulating devices, and allows to run flexibly configured but fully automated closed\sphinxhyphen{}loop Brain Stimulation protocols.

\sphinxAtStartPar
BEST Toolbox allows user to run customized Brain Stimulation experiments including basic measures of cortical excitability such as motor hotspot search, motor threshold hunting, motor evoked potential (MEP) and TMS\sphinxhyphen{}evoked EEG potential (TEP) measurements, estimation of stimulus\sphinxhyphen{}response curves, rTMS intervention protocols, etc., and since recently also Brain State\sphinxhyphen{}dependent or real\sphinxhyphen{}time EEG\sphinxhyphen{}triggered stimulation.

\sphinxAtStartPar
Along with its many technical abilities, the toolbox features a state\sphinxhyphen{}of\sphinxhyphen{}the\sphinxhyphen{}art and flexible MATLAB based application \textendash{} a Graphical User Interface to easily design experiments, online interactions with the data, visualization of data and a standardized format for the data under collection.


\chapter{Hardware Interfaces}
\label{\detokenize{index:hardware-interfaces}}
\sphinxAtStartPar
BEST Toolbox is currently optimized for bossdevice  (\sphinxhref{https://sync2brain.com}{sync2brain}) , a data processing and control system implemented as Simulink© Real\sphinxhyphen{}Time model on a high performance computer system receiving a digital real\sphinxhyphen{}time data stream from an EEG system such such as:
\begin{itemize}
\item {} 
\sphinxAtStartPar
NeurOne TESLA (Bittium, FL)

\item {} 
\sphinxAtStartPar
actiCHamp Plus (BrainProducts, DE)

\item {} 
\sphinxAtStartPar
CED 1400 (Power and Micro)

\end{itemize}

\sphinxAtStartPar
Additionaly a native implementation of following buffers is also part of the toolbox.
\begin{itemize}
\item {} 
\sphinxAtStartPar
FieldTrip Real\sphinxhyphen{}Time Buffer

\end{itemize}


\section{Input Devices}
\label{\detokenize{index:input-devices}}
\sphinxAtStartPar
The bossdevice and FieldTrip real\sphinxhyphen{}time buffer in turn allows the BEST Toolbox to interface with a wide variety of hardware and streaming platforms. Including but not limited to followings:
\begin{itemize}
\item {} 
\sphinxAtStartPar
Java

\item {} 
\sphinxAtStartPar
Python

\item {} 
\sphinxAtStartPar
Arduino

\item {} 
\sphinxAtStartPar
BCI2000 includes the FieldTripBuffer and the FieldTripBufferSource modules

\item {} 
\sphinxAtStartPar
BrainVision

\item {} 
\sphinxAtStartPar
NeurOne TESLA

\item {} 
\sphinxAtStartPar
BrainStream

\item {} 
\sphinxAtStartPar
ANT NeuroSDK

\item {} 
\sphinxAtStartPar
Artinis Medical Systems (NIRS)

\item {} 
\sphinxAtStartPar
BrainVision Recorder

\item {} 
\sphinxAtStartPar
Biosemi

\item {} 
\sphinxAtStartPar
CTF (MEG)

\item {} 
\sphinxAtStartPar
Emotiv

\item {} 
\sphinxAtStartPar
Neuromag/Elekta (MEG)

\item {} 
\sphinxAtStartPar
Jinga\sphinxhyphen{}Hi (LFP/EEG)

\item {} 
\sphinxAtStartPar
Micromed (ECoG)

\item {} 
\sphinxAtStartPar
ModularEEG/OpenEEG

\item {} 
\sphinxAtStartPar
Neuralynx (LFP)

\item {} 
\sphinxAtStartPar
Neurosky ThinkCap

\item {} 
\sphinxAtStartPar
OpenBCI

\item {} 
\sphinxAtStartPar
TMSI

\item {} 
\sphinxAtStartPar
TOBI

\end{itemize}

\sphinxAtStartPar
The details about implementation of FieldTrip real\sphinxhyphen{}time buffer can be found \sphinxhref{https://www.fieldtriptoolbox.org/development/realtime/implementation/}{here}.


\section{Output Devices}
\label{\detokenize{index:output-devices}}
\sphinxAtStartPar
BEST Toolbox is integrated with MAGIC toolbox in order to control and interact directly with the TMS devices that accepts TTL input for triggering and features API to set the device parameters:
\begin{itemize}
\item {} 
\sphinxAtStartPar
MagVenture

\item {} 
\sphinxAtStartPar
MagStim

\item {} 
\sphinxAtStartPar
BiStim

\item {} 
\sphinxAtStartPar
Rapid

\item {} 
\sphinxAtStartPar
DuoMag

\end{itemize}

\sphinxAtStartPar
In addition, it can also trigger any stimulation devices that can receive a TTL input trigger.


\subsection{Introduction to BEST Toolbox}
\label{\detokenize{1_Home:introduction-to-best-toolbox}}\label{\detokenize{1_Home::doc}}
\sphinxAtStartPar
\sphinxstylestrong{B}rain \sphinxstylestrong{E}lectrophysiological recording and \sphinxstylestrong{ST}imulation (\sphinxstylestrong{BEST}) Toolbox, is a MATLAB based open source software that interfaces with a wide variety of EEG, EMG, TMS and other stimulating devices, and allows to run flexibly configured but fully automated closed\sphinxhyphen{}loop Brain Stimulation protocols.

\sphinxAtStartPar
BEST Toolbox allows user to run customized Brain Stimulation experiments including basic measures of cortical excitability such as motor hotspot search, motor threshold hunting, motor evoked potential (MEP) and TMS\sphinxhyphen{}evoked EEG potential (TEP) measurements, estimation of stimulus\sphinxhyphen{}response curves, rTMS intervention protocols, etc., and since recently also Brain State\sphinxhyphen{}dependent or real\sphinxhyphen{}time EEG\sphinxhyphen{}triggered stimulation.

\sphinxAtStartPar
Along with its many technical abilities, the toolbox features a state\sphinxhyphen{}of\sphinxhyphen{}the\sphinxhyphen{}art and flexible MATLAB based application \textendash{} a Graphical User Interface to easily design experiments, online interactions with the data, visualization of data and a standardized format for the data under collection.


\subsubsection{Hardware Interfaces}
\label{\detokenize{1_Home:hardware-interfaces}}
\sphinxAtStartPar
BEST Toolbox is currently optimized for bossdevice  (\sphinxhref{https://sync2brain.com}{sync2brain}) , a data processing and control system implemented as Simulink© Real\sphinxhyphen{}Time model on a high performance computer system receiving a digital real\sphinxhyphen{}time data stream from an EEG system such such as:
\begin{itemize}
\item {} 
\sphinxAtStartPar
NeurOne TESLA (Bittium, FL)

\item {} 
\sphinxAtStartPar
actiCHamp Plus (BrainProducts, DE)

\item {} 
\sphinxAtStartPar
CED 1400 (Power and Micro)

\end{itemize}

\sphinxAtStartPar
Additionaly a native implementation of following buffers is also part of the toolbox.
\begin{itemize}
\item {} 
\sphinxAtStartPar
FieldTrip Real\sphinxhyphen{}Time Buffer

\end{itemize}


\paragraph{Input Devices}
\label{\detokenize{1_Home:input-devices}}
\sphinxAtStartPar
The bossdevice and FieldTrip real\sphinxhyphen{}time buffer in turn allows the BEST Toolbox to interface with a wide variety of hardware and streaming platforms. Including but not limited to followings:
\begin{itemize}
\item {} 
\sphinxAtStartPar
Java

\item {} 
\sphinxAtStartPar
Python

\item {} 
\sphinxAtStartPar
Arduino

\item {} 
\sphinxAtStartPar
BCI2000 includes the FieldTripBuffer and the FieldTripBufferSource modules

\item {} 
\sphinxAtStartPar
BrainVision

\item {} 
\sphinxAtStartPar
NeurOne TESLA

\item {} 
\sphinxAtStartPar
BrainStream

\item {} 
\sphinxAtStartPar
ANT NeuroSDK

\item {} 
\sphinxAtStartPar
Artinis Medical Systems (NIRS)

\item {} 
\sphinxAtStartPar
BrainVision Recorder

\item {} 
\sphinxAtStartPar
Biosemi

\item {} 
\sphinxAtStartPar
CTF (MEG)

\item {} 
\sphinxAtStartPar
Emotiv

\item {} 
\sphinxAtStartPar
Neuromag/Elekta (MEG)

\item {} 
\sphinxAtStartPar
Jinga\sphinxhyphen{}Hi (LFP/EEG)

\item {} 
\sphinxAtStartPar
Micromed (ECoG)

\item {} 
\sphinxAtStartPar
ModularEEG/OpenEEG

\item {} 
\sphinxAtStartPar
Neuralynx (LFP)

\item {} 
\sphinxAtStartPar
Neurosky ThinkCap

\item {} 
\sphinxAtStartPar
OpenBCI

\item {} 
\sphinxAtStartPar
TMSI

\item {} 
\sphinxAtStartPar
TOBI

\end{itemize}

\sphinxAtStartPar
The details about implementation of FieldTrip real\sphinxhyphen{}time buffer can be found \sphinxhref{https://www.fieldtriptoolbox.org/development/realtime/implementation/}{here}.


\paragraph{Output Devices}
\label{\detokenize{1_Home:output-devices}}
\sphinxAtStartPar
BEST Toolbox is integrated with MAGIC toolbox in order to control and interact directly with the TMS devices that accepts TTL input for triggering and features API to set the device parameters:
\begin{itemize}
\item {} 
\sphinxAtStartPar
MagVenture

\item {} 
\sphinxAtStartPar
MagStim

\item {} 
\sphinxAtStartPar
BiStim

\item {} 
\sphinxAtStartPar
Rapid

\item {} 
\sphinxAtStartPar
DuoMag

\end{itemize}

\sphinxAtStartPar
In addition, it can also trigger any stimulation devices that can receive a TTL input trigger.


\subsection{Download \& Setup}
\label{\detokenize{2_DownloadAndSetup:download-setup}}\label{\detokenize{2_DownloadAndSetup::doc}}

\subsubsection{MATLAB Version}
\label{\detokenize{2_DownloadAndSetup:matlab-version}}
\sphinxAtStartPar
The BEST Toolbox application is compatible with any MATLAB version older than r2006, however MATLAB version should exactly be r2017b when the BEST Toolbox is to be used in conjunction with the bossdevice.


\paragraph{Download BEST Toolbox Repository}
\label{\detokenize{2_DownloadAndSetup:download-best-toolbox-repository}}
\sphinxAtStartPar
The latest repository of BEST Toolbox prerelase version 0.1 can be downloaded from \sphinxhref{https://github.com/umair-hassan/BEST-Toolbox/tree/pre-release-0.1}{here}


\paragraph{Required APIs \& MATLAB Toolboxes}
\label{\detokenize{2_DownloadAndSetup:required-apis-matlab-toolboxes}}
\sphinxAtStartPar
BEST Toolbox is dependent on following APIs and MATLAB Toolboxes:
\begin{itemize}
\item {} 
\sphinxAtStartPar
\sphinxhref{https://api.sync2brain.com/}{Bossdevice API}  (also part of BEST Toolbox repository \textendash{} no download required)

\item {} 
\sphinxAtStartPar
\sphinxhref{https://github.com/nigelrogasch/MAGIC}{MAGIC}  (also part of BEST Toolbox repository \textendash{} no download required)

\item {} 
\sphinxAtStartPar
FieldTrip Toolbox (download required)

\end{itemize}

\sphinxAtStartPar
Additionally if the BEST Toolbox is intended to be used with bossdevice then please setup the bossdevice API as instructed on its website \sphinxhref{https://api.sync2brain.com/}{here}.  Generally, this will require following MathWorks products:
\begin{itemize}
\item {} 
\sphinxAtStartPar
Simulink Real\sphinxhyphen{}Time

\item {} 
\sphinxAtStartPar
Simulink Coder

\end{itemize}


\paragraph{Setup}
\label{\detokenize{2_DownloadAndSetup:setup}}

\subparagraph{BEST Toolbox}
\label{\detokenize{2_DownloadAndSetup:best-toolbox}}
\sphinxAtStartPar
After downloading BEST Toolbox from the branch as instructed above, perform the following steps:

\sphinxAtStartPar
1.Unzip the repository’s zip file

\sphinxAtStartPar
2.Copy the path of unzipped repository

\sphinxAtStartPar
3.Use the following command syntax on your MATLAB Command Window after replacing \textless{}path\_to\_besttoolbox\_downloaded\_UNZipped\_repository\textgreater{} with the path copied in 2nd step

\begin{sphinxVerbatim}[commandchars=\\\{\}]
\PYG{n+nb}{addpath}\PYG{p}{(}\PYG{n+nb}{genpath}\PYG{p}{(}\PYG{l+s}{\PYGZsq{}}\PYG{l+s}{\PYGZlt{}path\PYGZus{}to\PYGZus{}besttoolbox\PYGZus{}downloaded\PYGZus{}UNZipped\PYGZus{}repository\PYGZgt{}\PYGZsq{}}\PYG{p}{)}\PYG{p}{)}\PYG{p}{;}
\PYG{n+nb}{savepath}\PYG{p}{;}
\end{sphinxVerbatim}


\subparagraph{FieldTrip Toolbox}
\label{\detokenize{2_DownloadAndSetup:fieldtrip-toolbox}}
\sphinxAtStartPar
In order to setup FieldTrip properly please follow the tutorial given \sphinxhref{https://www.fieldtriptoolbox.org/faq/should\_i\_add\_fieldtrip\_with\_all\_subdirectories\_to\_my\_matlab\_path/}{here}


\subsection{Start BEST Toolbox Application}
\label{\detokenize{3_StartBESTToolboxApplication:start-best-toolbox-application}}\label{\detokenize{3_StartBESTToolboxApplication::doc}}
\sphinxAtStartPar
After the BEST Toolbox repository has been downloaded and setup, go to the MATLAB command window and execute the following command:

\begin{sphinxVerbatim}[commandchars=\\\{\}]
\PYG{n}{BEST}
\end{sphinxVerbatim}

\sphinxAtStartPar
As a result of above execution, a Graphical User Interface will pop up on the screen such as shown in Figure below.

\begin{figure}[htbp]
\centering

\noindent\sphinxincludegraphics{{fig1_startbesttoolboxapplication}.png}
\end{figure}


\subsection{Design Experiment}
\label{\detokenize{4_DesignExperiment:design-experiment}}\label{\detokenize{4_DesignExperiment::doc}}\begin{itemize}
\item {} 
\sphinxAtStartPar
Designing an Experiment consists of 4 fundamental steps:

\item {} 
\sphinxAtStartPar
Defining Experiment Title, Subject Code

\item {} 
\sphinxAtStartPar
Creating Session(s) for Experiment

\item {} 
\sphinxAtStartPar
Creating Protocol(s) for each Session

\end{itemize}

\sphinxAtStartPar
All the aspects of Designing an Experiment are mainly dealt in the following view of the application except the Protocol’s related parameters.

\begin{figure}[htbp]
\centering

\noindent\sphinxincludegraphics{{fig2_design_experiment1}.png}
\end{figure}


\subsubsection{Adding Experiment \& Subject}
\label{\detokenize{4_DesignExperiment:adding-experiment-subject}}
\sphinxAtStartPar
Once the BEST Toolbox application has been launched the next steps are to add the Experiment Title and Subject Code in the text fields available on the GUI. The title you will enter here will be used as a reference to store the data in your current directory for this particular experiment. The stored data can also be populated again in a fresh BEST Toolbox app window by just click the “Upload” button available in the menu and navigating it to that particular file.

\sphinxAtStartPar
Note that the Experiment Title and Subject Codes are both strings that your Operating System (OS) would allow as a name of folder or file, otherwise you are prompted to change the names by BEST Toolbox.


\subsubsection{Adding Sessions}
\label{\detokenize{4_DesignExperiment:adding-sessions}}
\sphinxAtStartPar
In order to add the session, just type the name of the session which could be any meaningful string and click on the + button on the right to the “Session Title” field so that it will get entered in a list named as “Sessions” in the application.

\sphinxAtStartPar
\sphinxhyphen{}Using the same procedure you can add as many sessions as required.

\sphinxAtStartPar
\sphinxhyphen{}Right click on the sessions’s list box would allow you to:
\begin{itemize}
\item {} 
\sphinxAtStartPar
Delete

\item {} 
\sphinxAtStartPar
Copy

\item {} 
\sphinxAtStartPar
Paste

\item {} 
\sphinxAtStartPar
Moveup

\item {} 
\sphinxAtStartPar
Movedown

\end{itemize}

\sphinxAtStartPar
any particular session.

\sphinxAtStartPar
\sphinxhyphen{}BEST Toolbox has integrated session management in its workflow. This reason behind this module of session management is to keep the data per Experiment as much consolidated and disciplined as possible. For an instance, if an experiment requires the same subject to undergo through three different sessions (either on different days or different kind of interventions), then with this session management tool of BEST Toolbox, you can just design the session only once and copy, paste that to repeat the exactly same steps.


\subsubsection{Adding Protocol}
\label{\detokenize{4_DesignExperiment:adding-protocol}}
\sphinxAtStartPar
In order to add the Protocols under a session (which are basically all the basic functions such as MEP measurement etc), just click on the drop\sphinxhyphen{}down menu next to the “Select Protocol” text and then select any Protocol you want to perform. Currently the available measures are:


\begin{savenotes}\sphinxattablestart
\centering
\sphinxcapstartof{table}
\sphinxthecaptionisattop
\sphinxcaption{Protocols in the BEST Toolbox}\label{\detokenize{4_DesignExperiment:id1}}
\sphinxaftertopcaption
\begin{tabulary}{\linewidth}[t]{|T|}
\hline

\sphinxAtStartPar
MEP Hotspot Search
\\
\hline
\sphinxAtStartPar
MEP Threshold Hunting
\\
\hline
\sphinxAtStartPar
MEP Dose Response Curve
\\
\hline
\sphinxAtStartPar
MEP Measurement
\\
\hline
\sphinxAtStartPar
rsEEG Measurement
\\
\hline
\sphinxAtStartPar
Psychometric Threshold Hunting
\\
\hline
\sphinxAtStartPar
TEP Hotspot Search
\\
\hline
\sphinxAtStartPar
TEP Measurement
\\
\hline
\sphinxAtStartPar
rTMS Intervention
\\
\hline
\sphinxAtStartPar
TMS fMRI Measurement
\\
\hline
\end{tabulary}
\par
\sphinxattableend\end{savenotes}

\sphinxAtStartPar
Then click on the + button, in order to add Protocol and further modify its Parameters and run it.

\sphinxAtStartPar
\sphinxhyphen{}Using the same procedure you can add as many Protocols as required.

\sphinxAtStartPar
\sphinxhyphen{}Right click on the Protocol’s list would allow you to:
\begin{itemize}
\item {} 
\sphinxAtStartPar
Delete

\item {} 
\sphinxAtStartPar
Copy

\item {} 
\sphinxAtStartPar
Paste

\item {} 
\sphinxAtStartPar
Moveup

\item {} 
\sphinxAtStartPar
Movedown

\item {} 
\sphinxAtStartPar
Load Results (if previous results are available)

\end{itemize}

\sphinxAtStartPar
any particular Protocol.

\sphinxAtStartPar
Status of each Protocol is also provided in the list next to the list of Protocols. Following statuses are possible:
\begin{itemize}
\item {} 
\sphinxAtStartPar
Created

\item {} 
\sphinxAtStartPar
Compiled

\item {} 
\sphinxAtStartPar
Successful | Starting Data Time Stamp | Ending Date Time Stamp

\item {} 
\sphinxAtStartPar
Stopped | Starting Data Time Stamp | Ending Date Time Stamp

\item {} 
\sphinxAtStartPar
Error | Starting Data Time Stamp | Ending Date Time Stamp

\end{itemize}


\subsubsection{Summary}
\label{\detokenize{4_DesignExperiment:summary}}
\sphinxAtStartPar
Once the Experiment, Subject Code, Sessions and Protocols are populated the application would get an appearance similar to Figure below. The selected Session and Protocol are highlighted and their stimulation or experimental parameters to be filled in by user are shown on the also to the right termed as “Protocol Designer”. Dedicated documentation pages are available for instructions to fill out the stimulation or experimental parameters in this wiki.

\begin{figure}[htbp]
\centering

\noindent\sphinxincludegraphics{{fig3_design_experiment_summary2}.png}
\end{figure}


\subsection{Hardware Configuration}
\label{\detokenize{5_HardwareConfiguration:hardware-configuration}}\label{\detokenize{5_HardwareConfiguration::doc}}

\subsubsection{Launching Hardware Configuration Panel}
\label{\detokenize{5_HardwareConfiguration:launching-hardware-configuration-panel}}
\sphinxAtStartPar
An “Open Hardware Config” named button is available on the main menu of GUI application of BEST Toolbox. Clicking it would lead you towards the Hardware Configuration area comprising of following view:


\subsubsection{Adding Output Devices (Host PC Controlled)}
\label{\detokenize{5_HardwareConfiguration:adding-output-devices-host-pc-controlled}}
\sphinxAtStartPar
You can allow BEST Toolbox to use your host PC to control the stimulation devices (output devices). Use the following pairs of parameters in order to allow so:
\begin{enumerate}
\sphinxsetlistlabels{\arabic}{enumi}{enumii}{}{.}%
\item {} 
\sphinxAtStartPar
Device Type: Select Output Device from drop down menu

\item {} 
\sphinxAtStartPar
Select Device: Select your device manufacturer from the given list of devices

\item {} 
\sphinxAtStartPar
Device Reference Name: Give any meaningful reference name to your device

\item {} 
\sphinxAtStartPar
COM Port Address: Connect the stimulation device to your Host PC using a serial COM cable and give the address of COM port here e.g. ‘COM1’

\item {} 
\sphinxAtStartPar
Add Button: Click on the add button at the bottom and the device will be shown on the left list box

\end{enumerate}

\sphinxAtStartPar
The overall populated panel in this case would look like as shown in figure below.


\subsubsection{Adding Output Devices (bossdevice Controlled)}
\label{\detokenize{5_HardwareConfiguration:adding-output-devices-bossdevice-controlled}}
\sphinxAtStartPar
You can allow BEST Toolbox to use your bossdevice to control the stimulation devices (output devices). Use the following pairs of parameters in order to allow so:
\begin{enumerate}
\sphinxsetlistlabels{\arabic}{enumi}{enumii}{}{.}%
\item {} 
\sphinxAtStartPar
Device Type: Select Output Device from drop down menu

\item {} 
\sphinxAtStartPar
Select Device: Select your device manufacturer from the given list of devices

\item {} 
\sphinxAtStartPar
Device Reference Name: Give any meaningful reference name to your device

\item {} 
\sphinxAtStartPar
COM Port Address: Connect the stimulation device to your Host PC using a serial COM cable and give the address of COM port here e.g. ‘COM1’ BOSS Box Output Port: Port number that is connecting the Boss Box trigger output to your stimulation device trigger input e.g. 1 BOSS Box Input Port: Port number that is connecting the Boss Box trigger input to your stimulation device trigger output e.g. 1

\item {} 
\sphinxAtStartPar
Add Button: Click on the add button at the bottom and the device will be shown on the left list box

\end{enumerate}

\sphinxAtStartPar
The overall populated panel in this case would look like as shown in figure below.


\subsubsection{Adding Output Devices (Digitimer)}
\label{\detokenize{5_HardwareConfiguration:adding-output-devices-digitimer}}
\sphinxAtStartPar
You can allow BEST Toolbox to control Digitimer using Arduino or Manually using Keyboard responses. Use the following pairs of parameters in order to allow so:
\begin{enumerate}
\sphinxsetlistlabels{\arabic}{enumi}{enumii}{}{.}%
\item {} 
\sphinxAtStartPar
Device Type: Select Output Device from drop down menu

\item {} 
\sphinxAtStartPar
Select Device: Select your device manufacturer from the given list of devices, in this case Digitimer

\item {} 
\sphinxAtStartPar
Device Reference Name: Give any meaningful reference name to your device

\item {} 
\sphinxAtStartPar
Trigger Control: Select from the given options of , bossdevice, Host PC Serial PCI Card, Host PC Parallel PCI Card, Arduino, Raspberry Pi, Manual and fill the subsequent relevant fields

\item {} 
\sphinxAtStartPar
Intensity Control: Select from the given options of, Arduino or Manual and fill the relevant subsequent relevant fields

\item {} 
\sphinxAtStartPar
Add Button: Click on the add button at the bottom and the device will be shown on the left list box

\end{enumerate}

\sphinxAtStartPar
The overall populated panel in this case would look like as shown in figure below.

\begin{figure}[htbp]
\centering

\noindent\sphinxincludegraphics{{fig4_adding_output_Devices}.png}
\end{figure}


\subsubsection{Adding Input Devices (BOSS Box Controlled)}
\label{\detokenize{5_HardwareConfiguration:adding-input-devices-boss-box-controlled}}
\sphinxAtStartPar
You can allow BEST Toolbox to use your BOSS Box to take input from the recording system (input devices). Use the following pairs of parameters in order to allow so:
\begin{enumerate}
\sphinxsetlistlabels{\arabic}{enumi}{enumii}{}{.}%
\item {} 
\sphinxAtStartPar
Device Type: Select Input Device from drop down menu

\item {} 
\sphinxAtStartPar
Select Device: Select your device manufacturer from the given list of devices or FieldTrip buffer

\item {} 
\sphinxAtStartPar
Device Reference Name: Give any meaningful reference name to your device

\item {} 
\sphinxAtStartPar
Protocol File Name: Give the name of Protocol file available in MATLAB’s current directory. This file is used to extract the channel names being streamed from your recording device.

\item {} 
\sphinxAtStartPar
Add Button: Click on the add button at the bottom and the device will be shown on the left list box

\end{enumerate}

\sphinxAtStartPar
The overall populated panel in this case would look like as shown in figure below.


\subsubsection{Adding Input Devices (FieldTrip Buffer Controlled)}
\label{\detokenize{5_HardwareConfiguration:adding-input-devices-fieldtrip-buffer-controlled}}
\sphinxAtStartPar
You can allow BEST Toolbox to use your FieldTrip real\sphinxhyphen{}time Buffer to take input from the recording system (input devices). Use the following pairs of parameters in order to allow so:
\begin{enumerate}
\sphinxsetlistlabels{\arabic}{enumi}{enumii}{}{.}%
\item {} 
\sphinxAtStartPar
Device Type: Select Input Device from drop down menu

\item {} 
\sphinxAtStartPar
Select Device: Select your device manufacturer from the given list of devices or FieldTrip buffer

\item {} 
\sphinxAtStartPar
Device Reference Name: Give any meaningful reference name to your device

\item {} 
\sphinxAtStartPar
Hose Name: Type the Host name or IP address in use. e.g. localhost

\item {} 
\sphinxAtStartPar
Port Address: The port address used to stream the real\sphinxhyphen{}time buffer. e.g. 2222

\item {} 
\sphinxAtStartPar
Channel Labels: The channel labels can be typed in as a row vector e.g. 1:16 or \{‘A’,’B’,’C’\} or \{‘C3′,’FCz’\} etc

\item {} 
\sphinxAtStartPar
Block Size: Block size of a Field trip buffer e.g. 500 samples

\item {} 
\sphinxAtStartPar
Sampling Rate: Sampling rate in hertz e.g. 1000 Hz

\item {} 
\sphinxAtStartPar
Add Button: Click on the add button at the bottom and the device will be shown on the left list box

\end{enumerate}

\sphinxAtStartPar
The overall populated panel in this case would look like as shown in figure below.


\subsection{MEP Hotspot Searchh}
\label{\detokenize{6_MEPHotspotSearch:mep-hotspot-searchh}}\label{\detokenize{6_MEPHotspotSearch::doc}}
\sphinxAtStartPar
Motor Hotspot Search function of the BEST Toolbox, trigger the stimulating device on trial by trial basis in a given inter\sphinxhyphen{}trial\sphinxhyphen{}interval and presents you online results of MEPs in order to visualize the MEP shape and its amplitude stability at a given hotspot.


\subsubsection{Parameters Syntax}
\label{\detokenize{6_MEPHotspotSearch:parameters-syntax}}

\paragraph{Input Device}
\label{\detokenize{6_MEPHotspotSearch:input-device}}
\sphinxAtStartPar
Select the input device using drop down menu from previously added devices


\paragraph{Output Device}
\label{\detokenize{6_MEPHotspotSearch:output-device}}
\sphinxAtStartPar
Select the output device using drop down menu from previously added devices


\paragraph{Protocol Mode}
\label{\detokenize{6_MEPHotspotSearch:protocol-mode}}
\sphinxAtStartPar
Automated allows you to set an Inter Trial Interval for trigger control whereas upon selection of Manual mode, the Protocol


\paragraph{EMG Display Channels}
\label{\detokenize{6_MEPHotspotSearch:emg-display-channels}}
\sphinxAtStartPar
Type the channel name as a cell array in order to visualize its online results e.g. \{ ‘APBr’\}. Note that the channel name must resolve to the name in your streaming data.


\paragraph{EMG Extraction Period}
\label{\detokenize{6_MEPHotspotSearch:emg-extraction-period}}
\sphinxAtStartPar
{[}min max{]} in ms


\paragraph{EMG Display Period}
\label{\detokenize{6_MEPHotspotSearch:emg-display-period}}
\sphinxAtStartPar
{[}min max{]} in ms


\paragraph{MEP Search Window}
\label{\detokenize{6_MEPHotspotSearch:mep-search-window}}
\sphinxAtStartPar
Time window to look for the MEP P2P amplitude. {[}min max{]} in ms


\paragraph{No. of Trials}
\label{\detokenize{6_MEPHotspotSearch:no-of-trials}}
\sphinxAtStartPar
Total number of trials e.g. 100


\paragraph{Inter Trial Interval}
\label{\detokenize{6_MEPHotspotSearch:inter-trial-interval}}
\sphinxAtStartPar
ITI scalar, or a range in seconds e.g. 4 or {[}4 6{]}


\subsubsection{Starting the Protocol}
\label{\detokenize{6_MEPHotspotSearch:starting-the-protocol}}
\sphinxAtStartPar
To start Motor Hotspot Search Protocol, just press the “Run” button at the bottom of the “Experiment Controller”. The measurement can be stopped, paused/unpaused. In order to check if all the parameters have been setup correctly, pressing the “Compile” button would prompt the results of compiled code whether its good to go or not.

\sphinxAtStartPar
An instance of the filled stimulation parameters panel is shown below.

\begin{figure}[htbp]
\centering

\noindent\sphinxincludegraphics{{fig5_MEPHotspotStarting_the_Protocol}.png}
\end{figure}


\subsection{MEP Threshold Hunting}
\label{\detokenize{7_MEPThresholdHunting:mep-threshold-hunting}}\label{\detokenize{7_MEPThresholdHunting::doc}}
\sphinxAtStartPar
Motor Threshold Hunting function of the BEST Toolbox, trigger the stimulating device on trial by trial basis in a given inter\sphinxhyphen{}trial\sphinxhyphen{}interval and measures the Amplitude of MEP and then adapt the stimulation intensity for next trial on the basis of current MEP amplitude. It also presents you online results of MEPs and Stimulation Intensity traces in order to visualize the MEP shape and its threshold stability throughout the procedure. In its advanced form, Brain State\sphinxhyphen{} Dependent MEP Threshold using the parameters given below.


\subsubsection{Parameters Syntax}
\label{\detokenize{7_MEPThresholdHunting:parameters-syntax}}

\paragraph{Brain State}
\label{\detokenize{7_MEPThresholdHunting:brain-state}}
\sphinxAtStartPar
In Brain State Independent case, Inter Trial Interval controls the timing of the stimulus whereas in Brain State dependent case, real\sphinxhyphen{}time EEG analysis allows to track the ongoing Phase and Amplitude thresholds thereby allowing to determine specific Brain States such as mu\sphinxhyphen{}Rhythm Peak Phase etc. and then delivers the stimulus upon a parametric case match.


\paragraph{Input Device}
\label{\detokenize{7_MEPThresholdHunting:input-device}}
\sphinxAtStartPar
Select the input device using drop down menu from previously added devices


\paragraph{Inter Trial Interval}
\label{\detokenize{7_MEPThresholdHunting:inter-trial-interval}}
\sphinxAtStartPar
ITI scalar, or a range in seconds e.g. 4 or {[}4 6{]} or a cell array in order to create ITI based experimental conditions e.g. \{4,5,6\}


\paragraph{Threshold Method}
\label{\detokenize{7_MEPThresholdHunting:threshold-method}}
\sphinxAtStartPar
Select one of the two statistical threshold estimation methods that have been implemented:
\begin{itemize}
\item {} 
\sphinxAtStartPar
Adaptive Staircasing Estimation {[}1{]}

\item {} 
\sphinxAtStartPar
Maximum Likelihood Estimation {[}2{]} \textendash{} Dependent on MATLAB Statistics and Machine Learning Toolbox

\end{itemize}


\paragraph{Trials Per Condition}
\label{\detokenize{7_MEPThresholdHunting:trials-per-condition}}
\sphinxAtStartPar
Various conditions can be created using ITI, Oscillation Target Phase and/or Amplitude and using the interactive Stimulation Parameters Designer. This field applies to all the created conditions and should be a scalar number e.g. 10 or 20 etc.


\paragraph{EMG Display Channels}
\label{\detokenize{7_MEPThresholdHunting:emg-display-channels}}
\sphinxAtStartPar
Type the channel name as a cell array in order to visualize its online results e.g. \{ ‘APBr’\}. Note that the channel name must resolve to the name in your streaming data, however Display Channels does not contributes for Threshold Measurement, these channels are merely for visualization e.g. in case when neighboring muscles are also required to be monitored.


\paragraph{EMG Extraction Period}
\label{\detokenize{7_MEPThresholdHunting:emg-extraction-period}}
\sphinxAtStartPar
{[}min max{]} in ms


\paragraph{EMG Display Period}
\label{\detokenize{7_MEPThresholdHunting:emg-display-period}}
\sphinxAtStartPar
{[}min max{]} in ms


\paragraph{MEP Search Window}
\label{\detokenize{7_MEPThresholdHunting:mep-search-window}}
\sphinxAtStartPar
Time window to look for the MEP P2P amplitude. {[}min max{]} in ms


\paragraph{Trials to Average}
\label{\detokenize{7_MEPThresholdHunting:trials-to-average}}
\sphinxAtStartPar
Both of the threshold methods , average certain number of trials in order to estimate the final threshold value, this parameter is specified as a scalar e.g. 10 and can be updated in run time as well.


\paragraph{Real\sphinxhyphen{}Time Channels Montage}
\label{\detokenize{7_MEPThresholdHunting:real-time-channels-montage}}
\sphinxAtStartPar
1xN cell array of Channel Names being streamed from the bio signal processor e.g. \{ ‘C3’, ‘FC1’, ‘FC5’, ‘CP1’, ‘CP5’\}


\paragraph{Real\sphinxhyphen{}Time Channels Weights}
\label{\detokenize{7_MEPThresholdHunting:real-time-channels-weights}}
\sphinxAtStartPar
1xN Numeric array of weights indexed w.r.t. to Channels Montage explained above e.g. 1 \sphinxhyphen{}0.25 \sphinxhyphen{}0.25 \sphinxhyphen{}0.25 \sphinxhyphen{}0.25


\paragraph{Frequency Band}
\label{\detokenize{7_MEPThresholdHunting:frequency-band}}
\sphinxAtStartPar
Choose the respective frequency band from the dropdown (Hz).


\paragraph{Peak Frequency}
\label{\detokenize{7_MEPThresholdHunting:peak-frequency}}
\sphinxAtStartPar
Scalar Peak Frequency in Hz. In order to import it from created or successful rsEEG Measurement Protocol, select the respective rsEEG Measurement protocol from the adjacent dropdown menu.


\paragraph{Target Phase}
\label{\detokenize{7_MEPThresholdHunting:target-phase}}
\sphinxAtStartPar
1xN Numeric array of Phase angles in radians. This parameter also creates N experimental conditions crossed over with all the other experimental conditions. If columns in this parameter is balanced with the rows in Amplitude Threshold parameter, then balanced conditions are created otherwise these 2 parameters are also crossed over.


\paragraph{Phase Tolerance}
\label{\detokenize{7_MEPThresholdHunting:phase-tolerance}}
\sphinxAtStartPar
Scalar Tolerance value in radians. Defining absolute target phase angles in order to detect a brain state is often prone to error mainly due to the resolution of data obtained after sampling rate transition. In order to overcome this digitization resolution error another parameter has to be defined such that the vicinities of the target phase shall be made clear to the detection algorithm. For an instance, while detecting a 0 radians phase, the phase vector would probably look like this {[}\sphinxhyphen{}0.001324 \sphinxhyphen{}0.00234 0.00243 0.004324{]}, and since none of them are mathematically equivalent to zero therefore in order to not allow to skip such Oscillatory Peak events and to increase the accuracy of the phase detection, a tolerance value is to be provided.


\paragraph{Amplitude Threshold}
\label{\detokenize{7_MEPThresholdHunting:amplitude-threshold}}
\sphinxAtStartPar
Nx2 Numeric array of Amplitude Thresholds. The 2 column dimensions are minimum and maximum thresholds where as N (number of rows) creates N Amplitude Threshold conditions crossed over with all the other experimental conditions. If columns in this parameter is balanced with the rows in Target Phase parameter, then balanced conditions are created otherwise these 2 parameters are also crossed over. Units are selected from the drop\sphinxhyphen{}down adjacent to the parameter.


\paragraph{Amplitude Assignment Period}
\label{\detokenize{7_MEPThresholdHunting:amplitude-assignment-period}}
\sphinxAtStartPar
If the Amplitude Threshold units are percentile, then the percentile is calculated over a certain time period defined in this parameter. This parameter enables the Brain State detection algorithms to cope with the variations in amplitude of large scale oscillatory activity e.g. due to variations in background neuronal activity.


\paragraph{EEG Extraction Period}
\label{\detokenize{7_MEPThresholdHunting:eeg-extraction-period}}
\sphinxAtStartPar
{[}min max{]} in ms


\paragraph{EEG Display Period}
\label{\detokenize{7_MEPThresholdHunting:eeg-display-period}}
\sphinxAtStartPar
{[}min max{]} in ms


\subsubsection{Creating Conditions Using Stimulation Parameters Designer}
\label{\detokenize{7_MEPThresholdHunting:creating-conditions-using-stimulation-parameters-designer}}
\sphinxAtStartPar
The Target Channels and Stimulation Trigger pattern can be defined in an interactive Stimulation Parameters Designer comprising of a tabular and graphical view. Following video illustrates that how conditions can be created using the intuitive designer.


\subsubsection{Starting the Protocol}
\label{\detokenize{7_MEPThresholdHunting:starting-the-protocol}}
\sphinxAtStartPar
To start Motor Threshold Hunting Protocol, just press the “Run” button at the bottom of the “Experiment Controller”. The measurement can be stopped, paused/unpaused. In order to check if all the parameters have been setup correctly, pressing the “Compile” button would prompt the results of compiled code whether its good to go or not.

\sphinxAtStartPar
An instance of the filled stimulation parameters panel is shown below.

\begin{figure}[htbp]
\centering

\noindent\sphinxincludegraphics{{fig6_MEPThresholdstarting_the_protocol}.png}
\end{figure}


\subsubsection{References}
\label{\detokenize{7_MEPThresholdHunting:references}}\begin{enumerate}
\sphinxsetlistlabels{\arabic}{enumi}{enumii}{}{.}%
\item {} 
\sphinxAtStartPar
Taylor, Martin \& Creelman, Douglas. (1967). PEST: Efficient Estimates on Probability Functions. The Journal of the Acoustical Society of America. 41. 782\sphinxhyphen{}787. 10.1121/1.1910407.

\item {} 
\sphinxAtStartPar
Pentland, A. Maximum likelihood estimation: The best PEST. Perception \& Psychophysics 28, 377\textendash{}379 (1980). \sphinxurl{https://doi.org/10.3758/BF03204398}

\end{enumerate}


\subsection{MEP Dose Response Curve}
\label{\detokenize{8_MEPDoseResponseCurve:mep-dose-response-curve}}\label{\detokenize{8_MEPDoseResponseCurve::doc}}
\sphinxAtStartPar
MEP Dose Response Curves Protocol of the BEST Toolbox, trigger the stimulating device on trial by trial basis in a given inter\sphinxhyphen{}trial\sphinxhyphen{}interval against set of stimulation intensity conditions and presents you online results of MEPs in order to visualize the MEP shape and quality of data being collected as well as the scatter plot of the dose\sphinxhyphen{}response curve and a sigmoid fitted dose\sphinxhyphen{}response curve at the stop event or when all trials are completed. In its advanced form, Brain State\sphinxhyphen{} Dependent MEP Dose\sphinxhyphen{}Response Curve can also be configured using the parameters explained below.
The flexible stimulation condition designer also facilitates versatile Paired Pulse protocols such as Short Interval Intracortical Inhibition (SICI), Long Interval Intracortical Inhibition (LICI), Short Afferent Inhibition (SAI) etc.
Dependent on the Dose Function selected in the Parameters, various important results such as Inflection Point, Plateaus, Inhibition and Facilitation are also estimated from the resulting Dose\sphinxhyphen{}Response curve in run time and annotated on the results.


\subsubsection{Parameters Syntax}
\label{\detokenize{8_MEPDoseResponseCurve:parameters-syntax}}

\paragraph{Brain State}
\label{\detokenize{8_MEPDoseResponseCurve:brain-state}}
\sphinxAtStartPar
In Brain State Independent case, Inter Trial Interval controls the timing of the stimulus whereas in Brain State dependent case, real\sphinxhyphen{}time EEG analysis allows to track the ongoing Phase and Amplitude thresholds thereby allowing to determine specific Brain States such as mu\sphinxhyphen{}Rhythm Peak Phase etc. and then delivers the stimulus upon a parametric case match.


\paragraph{Input Device}
\label{\detokenize{8_MEPDoseResponseCurve:input-device}}
\sphinxAtStartPar
Select the input device using drop down menu from previously added devices


\paragraph{Inter Trial Interval}
\label{\detokenize{8_MEPDoseResponseCurve:inter-trial-interval}}
\sphinxAtStartPar
ITI scalar, or a range in seconds e.g. 4 or {[}4 6{]} or a cell array in order to create ITI based experimental conditions e.g. \{4,5,6\}


\paragraph{Trials Per Condition}
\label{\detokenize{8_MEPDoseResponseCurve:trials-per-condition}}
\sphinxAtStartPar
Various conditions can be created using ITI, Oscillation Target Phase and/or Amplitude and using the interactive Stimulation Parameters Designer. This field applies to all the created conditions and should be a scalar number e.g. 10 or 20 etc.


\paragraph{EMG Target Channels}
\label{\detokenize{8_MEPDoseResponseCurve:emg-target-channels}}
\sphinxAtStartPar
Type the channel names as a 1xN cell array in order to get a Dose\sphinxhyphen{}Response curve for e.g. \{ ‘APBr’,’FDIr’\}. Note that the channel name must resolve to the name in your streaming data. Each target channel is crossed with all the other conditions of ITI, Phase, Amplitude and stimulation conditions.


\paragraph{EMG Display Channels}
\label{\detokenize{8_MEPDoseResponseCurve:emg-display-channels}}
\sphinxAtStartPar
Type the channel name as a 1xN cell array in order to visualize its online results e.g. \{ ‘APBr’\}. Note that the channel name must resolve to the name in your streaming data, however Display Channels does not contributes for Threshold Measurement, these channels are merely for visualization e.g. in case when neighboring muscles are also required to be monitored.


\paragraph{EMG Extraction Period}
\label{\detokenize{8_MEPDoseResponseCurve:emg-extraction-period}}
\sphinxAtStartPar
{[}min max{]} in ms


\paragraph{EMG Display Period}
\label{\detokenize{8_MEPDoseResponseCurve:emg-display-period}}
\sphinxAtStartPar
{[}min max{]} in ms


\paragraph{MEP Search Window}
\label{\detokenize{8_MEPDoseResponseCurve:mep-search-window}}
\sphinxAtStartPar
Time window to look for the MEP P2P amplitude. {[}min max{]} in ms


\paragraph{Trials to Average}
\label{\detokenize{8_MEPDoseResponseCurve:trials-to-average}}
\sphinxAtStartPar
Both of the threshold methods , average certain number of trials in order to estimate the final threshold value, this parameter is specified as a scalar e.g. 10 and can be updated in run time as well.


\paragraph{Dose Function}
\label{\detokenize{8_MEPDoseResponseCurve:dose-function}}
\sphinxAtStartPar
Select the independent variable to be be treated as Dose in the Protocol. CS and ISI are useful Dose Functions while dealing with Paired\sphinxhyphen{}Pulse protocols.


\paragraph{Response Function}
\label{\detokenize{8_MEPDoseResponseCurve:response-function}}
\sphinxAtStartPar
In order to facilitate Paired\sphinxhyphen{}Pulse Inhibition and Facilitation protocols, the Response Function has been introduced, all the TS+CS conditions (the numerator of Response Function) are given (condition number as shown in the Table) in the first field and only TS conditions (the denominator of the Response function) are given in the second field. There should be only 1 condition number in the 2nd field i.e. only 1 TS only condition, or as much condition numbers in the 2nd field as in the first field i.e. each TS+CS Condition will be balanced by its respective TS only condition.

\sphinxAtStartPar
Note: In case when there is no Response Formula such as Single Pulse Dose\sphinxhyphen{}Response curve, used 1 in both the fields.

\sphinxAtStartPar
For example in case when there are 4 conditions in total and first three conditions (as per the Table’s column of Condition \#) are TS+CS conditions and the last one is TS only, then the first field of Response function would be {[}1 2 3{]} and the second field would be {[}4{]}.


\paragraph{Real\sphinxhyphen{}Time Channels Montage}
\label{\detokenize{8_MEPDoseResponseCurve:real-time-channels-montage}}
\sphinxAtStartPar
1xN cell array of Channel Names being streamed from the bio signal processor e.g. \{ ‘C3’, ‘FC1’, ‘FC5’, ‘CP1’, ‘CP5’\}


\paragraph{Real\sphinxhyphen{}Time Channels Weights}
\label{\detokenize{8_MEPDoseResponseCurve:real-time-channels-weights}}
\sphinxAtStartPar
1xN Numeric array of weights indexed w.r.t. to Channels Montage explained above e.g. 1 \sphinxhyphen{}0.25 \sphinxhyphen{}0.25 \sphinxhyphen{}0.25 \sphinxhyphen{}0.25


\paragraph{Frequency Band}
\label{\detokenize{8_MEPDoseResponseCurve:frequency-band}}
\sphinxAtStartPar
Choose the respective frequency band from the dropdown (Hz).


\paragraph{Peak Frequency}
\label{\detokenize{8_MEPDoseResponseCurve:peak-frequency}}
\sphinxAtStartPar
Scalar Peak Frequency in Hz. In order to import it from created or successful rsEEG Measurement Protocol, select the respective rsEEG Measurement protocol from the adjacent dropdown menu.


\paragraph{Target Phase}
\label{\detokenize{8_MEPDoseResponseCurve:target-phase}}
\sphinxAtStartPar
1xN Numeric array of Phase angles in radians. This parameter also creates N experimental conditions crossed over with all the other experimental conditions. If columns in this parameter is balanced with the rows in Amplitude Threshold parameter, then balanced conditions are created otherwise these 2 parameters are also crossed over.


\paragraph{Phase Tolerance}
\label{\detokenize{8_MEPDoseResponseCurve:phase-tolerance}}
\sphinxAtStartPar
Scalar Tolerance value in radians. Defining absolute target phase angles in order to detect a brain state is often prone to error mainly due to the resolution of data obtained after sampling rate transition. In order to overcome this digitization resolution error another parameter has to be defined such that the vicinities of the target phase shall be made clear to the detection algorithm. For an instance, while detecting a 0 radians phase, the phase vector would probably look like this {[}\sphinxhyphen{}0.001324 \sphinxhyphen{}0.00234 0.00243 0.004324{]}, and since none of them are mathematically equivalent to zero therefore in order to not allow to skip such Oscillatory Peak events and to increase the accuracy of the phase detection, a tolerance value is to be provided.


\paragraph{Amplitude Threshold}
\label{\detokenize{8_MEPDoseResponseCurve:amplitude-threshold}}
\sphinxAtStartPar
Nx2 Numeric array of Amplitude Thresholds. The 2 column dimensions are minimum and maximum thresholds where as N (number of rows) creates N Amplitude Threshold conditions crossed over with all the other experimental conditions. If columns in this parameter is balanced with the rows in Target Phase parameter, then balanced conditions are created otherwise these 2 parameters are also crossed over. Units are selected from the drop\sphinxhyphen{}down adjacent to the parameter.


\paragraph{Amplitude Assignment Period}
\label{\detokenize{8_MEPDoseResponseCurve:amplitude-assignment-period}}
\sphinxAtStartPar
If the Amplitude Threshold units are percentile, then the percentile is calculated over a certain time period defined in this parameter. This parameter enables the Brain State detection algorithms to cope with the variations in amplitude of large scale oscillatory activity e.g. due to variations in background neuronal activity.


\paragraph{EEG Extraction Period}
\label{\detokenize{8_MEPDoseResponseCurve:eeg-extraction-period}}
\sphinxAtStartPar
{[}min max{]} in ms


\paragraph{EEG Display Period}
\label{\detokenize{8_MEPDoseResponseCurve:eeg-display-period}}
\sphinxAtStartPar
{[}min max{]} in ms


\subsubsection{Creating Conditions Using Stimulation Parameters Designer}
\label{\detokenize{8_MEPDoseResponseCurve:creating-conditions-using-stimulation-parameters-designer}}
\sphinxAtStartPar
The Target Channels and Stimulation Trigger pattern can be defined in an interactive Stimulation Parameters Designer comprising of a tabular and graphical view. Following video illustrates that how conditions can be created using the intuitive designer. Note that the example below is associated to Motor Threshold Hunting however exactly same procedure applies for the Psychometric Threshold Hunting to create Threshold conditions.


\subsubsection{Starting the Protocol}
\label{\detokenize{8_MEPDoseResponseCurve:starting-the-protocol}}
\sphinxAtStartPar
To start this Protocol, just press the “Run” button at the bottom of the “Experiment Controller”. The measurement can be stopped, paused/unpaused. In order to check if all the parameters have been setup correctly, pressing the “Compile” button would prompt the results of compiled code whether its good to go or not.


\subsection{MEP Measurement}
\label{\detokenize{9_MEPMeasurement:mep-measurement}}\label{\detokenize{9_MEPMeasurement::doc}}
\sphinxAtStartPar
MEP Measurement Protocol of the BEST Toolbox, trigger the stimulating device on trial by trial basis in a given inter\sphinxhyphen{}trial\sphinxhyphen{}interval against set of stimulation intensity conditions and presents you online results of MEPs in order to visualize the MEP shape and quality of data being collected as well as MEP Peak to Peak amplitude. In its advanced form, Brain State\sphinxhyphen{} Dependent MEP Measurement can also be configured using the parameters explained below.


\subsubsection{Parameters Syntax}
\label{\detokenize{9_MEPMeasurement:parameters-syntax}}

\paragraph{Brain State}
\label{\detokenize{9_MEPMeasurement:brain-state}}
\sphinxAtStartPar
In Brain State Independent case, Inter Trial Interval controls the timing of the stimulus whereas in Brain State dependent case, real\sphinxhyphen{}time EEG analysis allows to track the ongoing Phase and Amplitude thresholds thereby allowing to determine specific Brain States such as mu\sphinxhyphen{}Rhythm Peak Phase etc. and then delivers the stimulus upon a parametric case match.


\paragraph{Input Device}
\label{\detokenize{9_MEPMeasurement:input-device}}
\sphinxAtStartPar
Select the input device using drop down menu from previously added devices


\paragraph{Inter Trial Interval}
\label{\detokenize{9_MEPMeasurement:inter-trial-interval}}
\sphinxAtStartPar
ITI scalar, or a range in seconds e.g. 4 or {[}4 6{]} or a cell array in order to create ITI based experimental conditions e.g. \{4,5,6\}


\paragraph{Trials Per Condition}
\label{\detokenize{9_MEPMeasurement:trials-per-condition}}
\sphinxAtStartPar
Various conditions can be created using ITI, Oscillation Target Phase and/or Amplitude and using the interactive Stimulation Parameters Designer. This field applies to all the created conditions and should be a scalar number e.g. 10 or 20 etc.


\paragraph{EMG Display Channels}
\label{\detokenize{9_MEPMeasurement:emg-display-channels}}
\sphinxAtStartPar
Type the channel name as a 1xN cell array in order to visualize its online results e.g. \{ ‘APBr’\}. Note that the channel name must resolve to the name in your streaming data, however Display Channels does not contributes for Threshold Measurement, these channels are merely for visualization e.g. in case when neighboring muscles are also required to be monitored.


\paragraph{EMG Extraction Period}
\label{\detokenize{9_MEPMeasurement:emg-extraction-period}}
\sphinxAtStartPar
{[}min max{]} in ms


\paragraph{EMG Display Period}
\label{\detokenize{9_MEPMeasurement:emg-display-period}}
\sphinxAtStartPar
{[}min max{]} in ms


\paragraph{MEP Search Window}
\label{\detokenize{9_MEPMeasurement:mep-search-window}}
\sphinxAtStartPar
Time window to look for the MEP P2P amplitude. {[}min max{]} in ms


\paragraph{Trials to Average}
\label{\detokenize{9_MEPMeasurement:trials-to-average}}
\sphinxAtStartPar
Both of the threshold methods , average certain number of trials in order to estimate the final threshold value, this parameter is specified as a scalar e.g. 10 and can be updated in run time as well.


\paragraph{Real\sphinxhyphen{}Time Channels Montage}
\label{\detokenize{9_MEPMeasurement:real-time-channels-montage}}
\sphinxAtStartPar
1xN cell array of Channel Names being streamed from the bio signal processor e.g. \{ ‘C3’, ‘FC1’, ‘FC5’, ‘CP1’, ‘CP5’\}


\paragraph{Real\sphinxhyphen{}Time Channels Weights}
\label{\detokenize{9_MEPMeasurement:real-time-channels-weights}}
\sphinxAtStartPar
1xN Numeric array of weights indexed w.r.t. to Channels Montage explained above e.g. 1 \sphinxhyphen{}0.25 \sphinxhyphen{}0.25 \sphinxhyphen{}0.25 \sphinxhyphen{}0.25


\paragraph{Frequency Band}
\label{\detokenize{9_MEPMeasurement:frequency-band}}
\sphinxAtStartPar
Choose the respective frequency band from the dropdown (Hz).


\paragraph{Peak Frequency}
\label{\detokenize{9_MEPMeasurement:peak-frequency}}
\sphinxAtStartPar
Scalar Peak Frequency in Hz. In order to import it from created or successful rsEEG Measurement Protocol, select the respective rsEEG Measurement protocol from the adjacent dropdown menu.


\paragraph{Target Phase}
\label{\detokenize{9_MEPMeasurement:target-phase}}
\sphinxAtStartPar
1xN Numeric array of Phase angles in radians. This parameter also creates N experimental conditions crossed over with all the other experimental conditions. If columns in this parameter is balanced with the rows in Amplitude Threshold parameter, then balanced conditions are created otherwise these 2 parameters are also crossed over.


\paragraph{Phase Tolerance}
\label{\detokenize{9_MEPMeasurement:phase-tolerance}}
\sphinxAtStartPar
Scalar Tolerance value in radians. Defining absolute target phase angles in order to detect a brain state is often prone to error mainly due to the resolution of data obtained after sampling rate transition. In order to overcome this digitization resolution error another parameter has to be defined such that the vicinities of the target phase shall be made clear to the detection algorithm. For an instance, while detecting a 0 radians phase, the phase vector would probably look like this {[}\sphinxhyphen{}0.001324 \sphinxhyphen{}0.00234 0.00243 0.004324{]}, and since none of them are mathematically equivalent to zero therefore in order to not allow to skip such Oscillatory Peak events and to increase the accuracy of the phase detection, a tolerance value is to be provided.


\paragraph{Amplitude Threshold}
\label{\detokenize{9_MEPMeasurement:amplitude-threshold}}
\sphinxAtStartPar
Nx2 Numeric array of Amplitude Thresholds. The 2 column dimensions are minimum and maximum thresholds where as N (number of rows) creates N Amplitude Threshold conditions crossed over with all the other experimental conditions. If columns in this parameter is balanced with the rows in Target Phase parameter, then balanced conditions are created otherwise these 2 parameters are also crossed over. Units are selected from the drop\sphinxhyphen{}down adjacent to the parameter.


\paragraph{Amplitude Assignment Period}
\label{\detokenize{9_MEPMeasurement:amplitude-assignment-period}}
\sphinxAtStartPar
If the Amplitude Threshold units are percentile, then the percentile is calculated over a certain time period defined in this parameter. This parameter enables the Brain State detection algorithms to cope with the variations in amplitude of large scale oscillatory activity e.g. due to variations in background neuronal activity.


\paragraph{EEG Extraction Period}
\label{\detokenize{9_MEPMeasurement:eeg-extraction-period}}
\sphinxAtStartPar
{[}min max{]} in ms


\paragraph{EEG Display Period}
\label{\detokenize{9_MEPMeasurement:eeg-display-period}}
\sphinxAtStartPar
{[}min max{]} in ms


\subsubsection{Creating Conditions Using Stimulation Parameters Designer}
\label{\detokenize{9_MEPMeasurement:creating-conditions-using-stimulation-parameters-designer}}
\sphinxAtStartPar
The Target Channels and Stimulation Trigger pattern can be defined in an interactive Stimulation Parameters Designer comprising of a tabular and graphical view. Following video illustrates that how conditions can be created using the intuitive designer. Note that the example below is associated to Motor Threshold Hunting however exactly same procedure applies for the MEP Measurement Protocol to create various measuring conditions.


\subsubsection{Starting the Protocol}
\label{\detokenize{9_MEPMeasurement:starting-the-protocol}}
\sphinxAtStartPar
To start this Protocol, just press the “Run” button at the bottom of the “Experiment Controller”. The measurement can be stopped, paused/unpaused. In order to check if all the parameters have been setup correctly, pressing the “Compile” button would prompt the results of compiled code whether its good to go or not.


\subsection{rsEEG Measurement}
\label{\detokenize{10_rsEEGMeasurement:rseeg-measurement}}\label{\detokenize{10_rsEEGMeasurement::doc}}
\sphinxAtStartPar
resting state EEG Measurement has been introduced in the BEST Toolbox in order to estimate individual’s EEG Peak Frequency so that the Brain State dependent protocols can be fed with this necessary information of personalizing the measurement to the subject. As an outcome the user obtains various Power spectrum and measures of Signal to Noise Ration (SNR) so that informed decision can be made about the Brain State dependent protocol.


\subsubsection{Parameters Syntax}
\label{\detokenize{10_rsEEGMeasurement:parameters-syntax}}

\paragraph{Spectral Analysis}
\label{\detokenize{10_rsEEGMeasurement:spectral-analysis}}
\sphinxAtStartPar
BEST Toolbox has been integrated with FieldTrip’s IRASA and Multi\sphinxhyphen{}tapered FFT frequency spectral analysis functions. The choice of one from the two has to be made here using a drop\sphinxhyphen{}down.


\paragraph{Input Device}
\label{\detokenize{10_rsEEGMeasurement:input-device}}
\sphinxAtStartPar
Select the output device using drop down menu from previously added devices


\paragraph{EEG Acquisition Period}
\label{\detokenize{10_rsEEGMeasurement:eeg-acquisition-period}}
\sphinxAtStartPar
The time period in minutes for which the resting state EEG is to be acquired and then the analysis to be applied immediately afterwards with no human involvement so that the Frequency results can be obtained immediately with in a few minutes. Its a scalar data type with units as minutes such as 5, means 5 minutes.


\paragraph{EEG Epoch Period}
\label{\detokenize{10_rsEEGMeasurement:eeg-epoch-period}}
\sphinxAtStartPar
The continuous data is epoch after acquisition and therefore the length of the epoch has been kept flexible. Its a scalar data type with units in seconds, such as 4.


\paragraph{Target Frequency Range}
\label{\detokenize{10_rsEEGMeasurement:target-frequency-range}}
\sphinxAtStartPar
{[}min max{]} in Hz, the individual peak frequency is determined with in this given range of frequency upon estimation of the power spectrum.


\paragraph{Montage Channels}
\label{\detokenize{10_rsEEGMeasurement:montage-channels}}
\sphinxAtStartPar
A cell array of strings containing Channel names being streamed, if a Montage has to be created, that is defined as a nested cell array of strings with all channel labels needed in that particular montage. For an instance \{‘C3’,\{ ‘C3’, ‘FC1’, ‘FC5’, ‘CP1’, ‘CP5’\}, ‘P3’\} , will provide 3 estimates of Peak Frequency, first one for the C3 channel, second one for the Montage \{ ‘C3’, ‘FC1’, ‘FC5’, ‘CP1’, ‘CP5’\} and third one for the channel P3. See Montage Weights for defining the weights of the montages.


\paragraph{Montage Weights}
\label{\detokenize{10_rsEEGMeasurement:montage-weights}}
\sphinxAtStartPar
A cell array of numeric indexed w.r.t to Montage Channels array. If Montage Channels contains a nested cell array, then the same structure would be applied to Montage weights as well with the only difference being the numeric weights instead of the channel names. For the example given in Montage Channels the weights can be \{1,\{1 ,\sphinxhyphen{}0.25, \sphinxhyphen{}0.25, \sphinxhyphen{}0.25, \sphinxhyphen{}0.25\},1\} , making the montage as Laplacian Hjorth Montage but the first and last channels being the simple ones. Similarly any kind of montage can be defined.


\paragraph{Reference Channels}
\label{\detokenize{10_rsEEGMeasurement:reference-channels}}
\sphinxAtStartPar
1xN cell array of channels name (strings) being streamed from the source, no rereferencing is applied if this field is empty and this field can also be ‘all’ for rereferencing against common average of all channels that are being streamed.


\paragraph{Recording Reference}
\label{\detokenize{10_rsEEGMeasurement:recording-reference}}
\sphinxAtStartPar
Implicit Reference channel label given as a string such as ‘FCz’, can be empty.


\paragraph{High Pass Frequency}
\label{\detokenize{10_rsEEGMeasurement:high-pass-frequency}}
\sphinxAtStartPar
1xN vector of High pass frequency in Hz for data pre processing, no high pass filter is applied if this field is left empty.


\paragraph{Band Stop Frequency}
\label{\detokenize{10_rsEEGMeasurement:band-stop-frequency}}
\sphinxAtStartPar
1xN vector of band stop frequencies in Hz for data pre processing, no band pass filter is applied if this field is left empty.


\subsubsection{Starting the Protocol}
\label{\detokenize{10_rsEEGMeasurement:starting-the-protocol}}
\sphinxAtStartPar
To start rsEEG Measurement Protocol, just press the “Run” button at the bottom of the “Experiment Controller”. The measurement can be stopped, however unlike other protocols cannot be paused/unpaused because that will cause discontinuities in data and makes the EEG analysis spurious. In order to check if all the parameters have been setup correctly, pressing the “Compile” button would prompt the results of compiled code whether its good to go or not.

\sphinxAtStartPar
An instance of the filled parameters panel is shown below.

\begin{figure}[htbp]
\centering

\noindent\sphinxincludegraphics{{fig7_rEEGstarting}.png}
\end{figure}


\subsection{TEP Measurement}
\label{\detokenize{11_TEPMeasurement:tep-measurement}}\label{\detokenize{11_TEPMeasurement::doc}}
\sphinxAtStartPar
The bossdevice research is a real\sphinxhyphen{}time digital signal processor consisting of hardware and software algorithms. It is designed to read\sphinxhyphen{}in a real\sphinxhyphen{}time raw data stream from a biosignal amplifier (electroencephalography, EEG), to continuously analyze the data and to detect patterns in this data based on oscillations in different frequencies. When such a pattern is detected, the device indicates this through a standard output port. This enables a stimulation device (such as a sound generator) to be triggered in response to a specific biosignal pattern occurring. The device can be programmed by the user to detect different patterns.

\begin{sphinxadmonition}{important}{Important:}
\sphinxAtStartPar
The bossdevice research is not a medical device. It may not be used outside of research and it may not be used in trials involving patients. It is not intended as an accessory to a medical device or to control a medical device. It may only be connected to a stimulation device if the stimulation device provides an input port for the purpose of receiving information regarding the desired timing of stimulation. Whether or not a stimulus is then generated in response to a signal from the bossdevice is determined by the stimulation device.
\end{sphinxadmonition}

\begin{figure}[htbp]
\centering
\capstart

\noindent\sphinxincludegraphics{{figures/Fig1_bossdeviceandneurone}.png}
\caption{The bossdevice research placed along with Bittim NeurOne biosignal amplifier.}\label{\detokenize{11_TEPMeasurement:id1}}\end{figure}


\subsection{TUS Intervention}
\label{\detokenize{13_TUSIntervention:tus-intervention}}\label{\detokenize{13_TUSIntervention::doc}}
\sphinxAtStartPar
The bossdevice research is a real\sphinxhyphen{}time digital signal processor consisting of hardware and software algorithms. It is designed to read\sphinxhyphen{}in a real\sphinxhyphen{}time raw data stream from a biosignal amplifier (electroencephalography, EEG), to continuously analyze the data and to detect patterns in this data based on oscillations in different frequencies. When such a pattern is detected, the device indicates this through a standard output port. This enables a stimulation device (such as a sound generator) to be triggered in response to a specific biosignal pattern occurring. The device can be programmed by the user to detect different patterns.

\begin{sphinxadmonition}{important}{Important:}
\sphinxAtStartPar
The bossdevice research is not a medical device. It may not be used outside of research and it may not be used in trials involving patients. It is not intended as an accessory to a medical device or to control a medical device. It may only be connected to a stimulation device if the stimulation device provides an input port for the purpose of receiving information regarding the desired timing of stimulation. Whether or not a stimulus is then generated in response to a signal from the bossdevice is determined by the stimulation device.
\end{sphinxadmonition}

\begin{figure}[htbp]
\centering
\capstart

\noindent\sphinxincludegraphics{{figures/Fig1_bossdeviceandneurone}.png}
\caption{The bossdevice research placed along with Bittim NeurOne biosignal amplifier.}\label{\detokenize{13_TUSIntervention:id1}}\end{figure}


\subsection{Introduction to bossdevice research}
\label{\detokenize{14_TMSfMRIMeasurement:introduction-to-bossdevice-research}}\label{\detokenize{14_TMSfMRIMeasurement::doc}}
\sphinxAtStartPar
The bossdevice research is a real\sphinxhyphen{}time digital signal processor consisting of hardware and software algorithms. It is designed to read\sphinxhyphen{}in a real\sphinxhyphen{}time raw data stream from a biosignal amplifier (electroencephalography, EEG), to continuously analyze the data and to detect patterns in this data based on oscillations in different frequencies. When such a pattern is detected, the device indicates this through a standard output port. This enables a stimulation device (such as a sound generator) to be triggered in response to a specific biosignal pattern occurring. The device can be programmed by the user to detect different patterns.

\begin{sphinxadmonition}{important}{Important:}
\sphinxAtStartPar
The bossdevice research is not a medical device. It may not be used outside of research and it may not be used in trials involving patients. It is not intended as an accessory to a medical device or to control a medical device. It may only be connected to a stimulation device if the stimulation device provides an input port for the purpose of receiving information regarding the desired timing of stimulation. Whether or not a stimulus is then generated in response to a signal from the bossdevice is determined by the stimulation device.
\end{sphinxadmonition}

\begin{figure}[htbp]
\centering
\capstart

\noindent\sphinxincludegraphics{{figures/Fig1_bossdeviceandneurone}.png}
\caption{The bossdevice research placed along with Bittim NeurOne biosignal amplifier.}\label{\detokenize{14_TMSfMRIMeasurement:id1}}\end{figure}


\subsection{Sensory Threshold Hunting}
\label{\detokenize{15_SensoryThresholdHunting:sensory-threshold-hunting}}\label{\detokenize{15_SensoryThresholdHunting::doc}}
\sphinxAtStartPar
Sensory Threshold Hunting function of the BEST Toolbox is mainly dealing for determination of sensory thresholds by triggering the stimulating device on trial by trial basis in a given inter\sphinxhyphen{}trial\sphinxhyphen{}interval and records the subject’s sensation feedback manual via a keyboard response of experimenter and then adapt the stimulation intensity for next trial on the basis of subject’s feedback via experimenter to the BEST Toolbox. It also presents you online results of Stimulation Intensity traces in order to visualize the threshold stability throughout the procedure. In its advanced form, Brain State\sphinxhyphen{} Dependent Sensory Threshold can also be performed using the parameters given below.


\subsubsection{Parameters Syntax}
\label{\detokenize{15_SensoryThresholdHunting:parameters-syntax}}

\paragraph{Brain State}
\label{\detokenize{15_SensoryThresholdHunting:brain-state}}
\sphinxAtStartPar
In Brain State Independent case, Inter Trial Interval controls the timing of the stimulus whereas in Brain State dependent case, real\sphinxhyphen{}time EEG analysis allows to track the ongoing Phase and Amplitude thresholds thereby allowing to determine specific Brain States such as mu\sphinxhyphen{}Rhythm Peak Phase etc. and then delivers the stimulus upon a parametric case match.


\paragraph{Input Device}
\label{\detokenize{15_SensoryThresholdHunting:input-device}}
\sphinxAtStartPar
Select the input device using drop down menu from previously added devices. Keyboard or a Response Button Pad can be added as input device.


\paragraph{Inter Trial Interval}
\label{\detokenize{15_SensoryThresholdHunting:inter-trial-interval}}
\sphinxAtStartPar
ITI scalar, or a range in seconds e.g. 4 or {[}4 6{]} or a cell array in order to create ITI based experimental conditions e.g. \{4,5,6\}


\paragraph{Threshold Method}
\label{\detokenize{15_SensoryThresholdHunting:threshold-method}}
\sphinxAtStartPar
Select one of the two statistical threshold estimation methods that have been implemented:
\begin{itemize}
\item {} 
\sphinxAtStartPar
Adaptive Staircasing Estimation {[}1{]}

\item {} 
\sphinxAtStartPar
Maximum Likelihood Estimation {[}2{]} \textendash{} Dependent on MATLAB Statistics and Machine Learning Toolbox

\end{itemize}


\paragraph{Trials Per Condition}
\label{\detokenize{15_SensoryThresholdHunting:trials-per-condition}}
\sphinxAtStartPar
Various conditions can be created using ITI, Oscillation Target Phase and/or Amplitude and using the interactive Stimulation Parameters Designer. This field applies to all the created conditions and should be a scalar number e.g. 10 or 20 etc.


\paragraph{Trials to Average}
\label{\detokenize{15_SensoryThresholdHunting:trials-to-average}}
\sphinxAtStartPar
Both of the threshold methods , average certain number of trials in order to estimate the final threshold value, this parameter is specified as a scalar e.g. 10 and can be updated in run time as well.


\paragraph{Real\sphinxhyphen{}Time Channels Montage}
\label{\detokenize{15_SensoryThresholdHunting:real-time-channels-montage}}
\sphinxAtStartPar
1xN cell array of Channel Names being streamed from the bio signal processor e.g. \{ ‘C3’, ‘FC1’, ‘FC5’, ‘CP1’, ‘CP5’\}


\paragraph{Real\sphinxhyphen{}Time Channels Weights}
\label{\detokenize{15_SensoryThresholdHunting:real-time-channels-weights}}
\sphinxAtStartPar
1xN Numeric array of weights indexed w.r.t. to Channels Montage explained above e.g. 1 \sphinxhyphen{}0.25 \sphinxhyphen{}0.25 \sphinxhyphen{}0.25 \sphinxhyphen{}0.25


\paragraph{Frequency Band}
\label{\detokenize{15_SensoryThresholdHunting:frequency-band}}
\sphinxAtStartPar
Choose the respective frequency band from the dropdown (Hz).


\paragraph{Peak Frequency}
\label{\detokenize{15_SensoryThresholdHunting:peak-frequency}}
\sphinxAtStartPar
Scalar Peak Frequency in Hz. In order to import it from created or successful rsEEG Measurement Protocol, select the respective rsEEG Measurement protocol from the adjacent dropdown menu.


\paragraph{Target Phase}
\label{\detokenize{15_SensoryThresholdHunting:target-phase}}
\sphinxAtStartPar
1xN Numeric array of Phase angles in radians. This parameter also creates N experimental conditions crossed over with all the other experimental conditions. If columns in this parameter is balanced with the rows in Amplitude Threshold parameter, then balanced conditions are created otherwise these 2 parameters are also crossed over.


\paragraph{Phase Tolerance}
\label{\detokenize{15_SensoryThresholdHunting:phase-tolerance}}
\sphinxAtStartPar
Scalar Tolerance value in radians. Defining absolute target phase angles in order to detect a brain state is often prone to error mainly due to the resolution of data obtained after sampling rate transition. In order to overcome this digitization resolution error another parameter has to be defined such that the vicinities of the target phase shall be made clear to the detection algorithm. For an instance, while detecting a 0 radians phase, the phase vector would probably look like this {[}\sphinxhyphen{}0.001324 \sphinxhyphen{}0.00234 0.00243 0.004324{]}, and since none of them are mathematically equivalent to zero therefore in order to not allow to skip such Oscillatory Peak events and to increase the accuracy of the phase detection, a tolerance value is to be provided.


\paragraph{Amplitude Threshold}
\label{\detokenize{15_SensoryThresholdHunting:amplitude-threshold}}
\sphinxAtStartPar
Nx2 Numeric array of Amplitude Thresholds. The 2 column dimensions are minimum and maximum thresholds where as N (number of rows) creates N Amplitude Threshold conditions crossed over with all the other experimental conditions. If columns in this parameter is balanced with the rows in Target Phase parameter, then balanced conditions are created otherwise these 2 parameters are also crossed over. Units are selected from the drop\sphinxhyphen{}down adjacent to the parameter.


\paragraph{Amplitude Assignment Period}
\label{\detokenize{15_SensoryThresholdHunting:amplitude-assignment-period}}
\sphinxAtStartPar
If the Amplitude Threshold units are percentile, then the percentile is calculated over a certain time period defined in this parameter. This parameter enables the Brain State detection algorithms to cope with the variations in amplitude of large scale oscillatory activity e.g. due to variations in background neuronal activity.


\paragraph{EEG Extraction Period}
\label{\detokenize{15_SensoryThresholdHunting:eeg-extraction-period}}
\sphinxAtStartPar
{[}min max{]} in ms


\paragraph{EEG Display Period}
\label{\detokenize{15_SensoryThresholdHunting:eeg-display-period}}
\sphinxAtStartPar
{[}min max{]} in ms


\subsubsection{Creating Conditions Using Stimulation Parameters Designer}
\label{\detokenize{15_SensoryThresholdHunting:creating-conditions-using-stimulation-parameters-designer}}
\sphinxAtStartPar
The Target Channels and Stimulation Trigger pattern can be defined in an interactive Stimulation Parameters Designer comprising of a tabular and graphical view. Following video illustrates that how conditions can be created using the intuitive designer. Note that the example below is associated to Motor Threshold Hunting however exactly same procedure applies for the Sensory Threshold Hunting to create Threshold conditions.


\subsubsection{Starting the Protocol}
\label{\detokenize{15_SensoryThresholdHunting:starting-the-protocol}}
\sphinxAtStartPar
To start Sensory Threshold Hunting Protocol, just press the “Run” button at the bottom of the “Experiment Controller”. The measurement can be stopped, paused/unpaused. In order to check if all the parameters have been setup correctly, pressing the “Compile” button would prompt the results of compiled code whether its good to go or not.


\subsubsection{References}
\label{\detokenize{15_SensoryThresholdHunting:references}}\begin{enumerate}
\sphinxsetlistlabels{\arabic}{enumi}{enumii}{}{.}%
\item {} 
\sphinxAtStartPar
Taylor, Martin \& Creelman, Douglas. (1967). PEST: Efficient Estimates on Probability Functions. The Journal of the Acoustical Society of America. 41. 782\sphinxhyphen{}787. 10.1121/1.1910407.

\item {} 
\sphinxAtStartPar
Pentland, A. Maximum likelihood estimation: The best PEST. Perception \& Psychophysics 28, 377\textendash{}379 (1980). \sphinxurl{https://doi.org/10.3758/BF03204398}

\end{enumerate}


\subsection{Issues, Bugs and Requests}
\label{\detokenize{15_IssuesBugsRequests:issues-bugs-and-requests}}\label{\detokenize{15_IssuesBugsRequests::doc}}
\sphinxAtStartPar
You first have to sign up for an account or log in on GitHub, subsequently you can report the issue on the GitHub repositroty linked below:

\begin{sphinxVerbatim}[commandchars=\\\{\}]
\PYG{n}{https}\PYG{p}{:}\PYG{o}{/}\PYG{o}{/}\PYG{n}{github}\PYG{p}{.}\PYG{n}{com}\PYG{o}{/}\PYG{n}{umair}\PYG{o}{\PYGZhy{}}\PYG{n}{hassan}\PYG{o}{/}\PYG{n}{best}\PYG{o}{\PYGZhy{}}\PYG{n}{toolbox}\PYG{o}{/}\PYG{n}{issues}
\end{sphinxVerbatim}

\sphinxAtStartPar
We automatically will receive an email and will follow up and keep you updated; i.e., you will get an email through GitHub whenever someone works on your issue.


\bigskip\hrule\bigskip



\subsubsection{Issues \& Bugs Details}
\label{\detokenize{15_IssuesBugsRequests:issues-bugs-details}}
\sphinxAtStartPar
The easier it is for one of the developers to reproduce your bug, the more likely it is that we’ll fix the problem. Good bug reports include a small test script and the data (i.e. mat file) required to reproduce the bug.

\sphinxAtStartPar
Please create a small test script and a piece of data that are both as small and simple as possible to reproduce the problem. For example: a .mat file containing the bossdevice API class object and some screenshots can lead us to solve the issue.


\bigskip\hrule\bigskip



\subsubsection{Use Case Scenarios}
\label{\detokenize{15_IssuesBugsRequests:use-case-scenarios}}
\begin{sphinxadmonition}{important}{Important:}
\sphinxAtStartPar
Use cases, previous research papers that might have implemented those methods or a brief description about the feature requests can help our developers to create the smartest solution for you.
\end{sphinxadmonition}


\subsection{Workshops}
\label{\detokenize{16_Workshops:workshops}}\label{\detokenize{16_Workshops::doc}}
\sphinxAtStartPar
We are considering and/or concretely planning to have BEST Toolbox workshops/webinars in
\begin{itemize}
\item {} 
\sphinxAtStartPar
\sphinxhref{https://brainbox-initiative.com/webinars}{2021 (Jul), Brain Box Initiative Webinar Series}

\end{itemize}

\sphinxAtStartPar
Previous BEST Toolbox presentations took place at:
\begin{itemize}
\item {} 
\sphinxAtStartPar
\sphinxhref{https://www.elsevier.com/events/conferences/international-brain-stimulation-conference}{2020 (Nov), 3rd International Brain Stimulation Conference, 2020, Virtual Conference.}

\item {} 
\sphinxAtStartPar
\sphinxhref{https://brainbox-initiative.com/conference/2020}{2020 (Sep), Brainbox Initiative Conference 2020, Virtual Conference.}

\item {} 
\sphinxAtStartPar
\sphinxhref{https://www.humanbrainmapping.org/i4a/pages/index.cfm?pageid=3958}{2020 (Jun), OHBM 2020, Virtual Conference.}

\end{itemize}


\subsection{About Us}
\label{\detokenize{17_About Us:about-us}}\label{\detokenize{17_About Us::doc}}
\sphinxAtStartPar
BEST Toolbox is an open source project aiming to increase objectivity, reliability and reproducibility in non invasive brain stimulation research. It was created in December 2018, with a focus on closed\sphinxhyphen{}loop real\sphinxhyphen{}time brain state dependent brain stimulation experiment control, now extending its features to the various modalities of neuromodulation. BEST Toolbox is developed in the open on GitHub.


\subsubsection{Team Members}
\label{\detokenize{17_About Us:team-members}}\begin{itemize}
\item {} 
\sphinxAtStartPar
\sphinxhref{https://twitter.com/neuro\_engineer}{Umair Hassan}

\item {} 
\sphinxAtStartPar
\sphinxhref{http://tobergmann.de/}{Prof. Dr. Tile Ole Bergman}

\item {} 
\sphinxAtStartPar
\sphinxhref{https://sync2brain.com}{Dr. med. Christoph Zrenner}

\item {} 
\sphinxAtStartPar
Steven Pillen

\end{itemize}



\renewcommand{\indexname}{Index}
\printindex
\end{document}